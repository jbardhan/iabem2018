\documentclass[11pt,twoside]{article}

\usepackage[utf8]{inputenc}
\usepackage[a4paper,twoside,bindingoffset=1cm,head=15pt,body={17cm,25cm},top=2.5cm]{geometry}
\usepackage{amsmath,amssymb}
\usepackage{graphicx}
\usepackage{color}
\usepackage{cite}
\usepackage{ifthen}
\usepackage{titlesec}
\usepackage{fancyhdr}

\setlength{\columnsep}{3em}
\titleformat{\section}{\normalfont\bfseries}{\thesection}{1em}{}
\titlespacing*{\section}{0pt}{*1.5}{1ex}



\newcommand{\todo}[1]{\textcolor{red}{\#\#\# #1 \#\#\#}}

\newcounter{authorcounter}

\newcommand{\iabemauthorlist}{}
\newcommand{\iabemaddresslist}{}
\newcommand{\iabememail}{}
\newcommand{\iabemfootnotes}{}
\newcommand{\iabemauthorpre}{}

\def\theNumberTest#1{%
  \if\relax\detokenize\expandafter{\romannumeral-0#1}\relax
    true%
  \else
    false%
  \fi
}

\newcommand{\iabemspeaker}[3][noemail]{%
    \ifthenelse{\value{authorcounter} > 1}{%
      \renewcommand{\iabemauthorpre}{, }%
    }{%
      \renewcommand{\iabemauthorpre}{}%
    }%
    \ifthenelse{\equal{#1}{noemail}}{%
      \renewcommand{\iabemfootnotes}{}
    }{%    
      \renewcommand{\iabememail}{$^\ast$Email: #1}%
      \renewcommand{\iabemfootnotes}{, \ast}
    }%
    \ifthenelse{\equal{\theNumberTest{#3}}{true}}{%
      \edef\iabemauthorlist{\iabemauthorlist%
        \iabemauthorpre{}\underline{#2}$^{#3\iabemfootnotes%
        }$%
      }%
    }{%    
      \edef\iabemauthorlist{\iabemauthorlist%
        \iabemauthorpre\underline{#2}$^{\arabic{authorcounter}\iabemfootnotes%
        }$%
      }
      \edef\iabemaddresslist{\iabemaddresslist% 
        \par%
        $^{\arabic{authorcounter}}$#3%
      }%      
      \stepcounter{authorcounter}%
    }%
  \ignorespaces
}

\newcommand{\iabemauthor}[3][noemail]{%
    \ifthenelse{\value{authorcounter} > 1}{%
      \renewcommand{\iabemauthorpre}{, }%
    }{%
      \renewcommand{\iabemauthorpre}{}%
    }%
    \ifthenelse{\equal{#1}{noemail}}{%
      \renewcommand{\iabemfootnotes}{}
    }{%    
      \renewcommand{\iabememail}{$^\ast$Email: #1}%
      \renewcommand{\iabemfootnotes}{, \ast}
    }%
    \ifthenelse{\equal{\theNumberTest{#3}}{true}}{%
      \edef\iabemauthorlist{\iabemauthorlist%
        \iabemauthorpre{}#2$^{#3\iabemfootnotes%
        }$%
      }%
    }{% 
      \edef\iabemauthorlist{\iabemauthorlist%
         \iabemauthorpre{}#2$^{\arabic{authorcounter}\iabemfootnotes%
         }$%
      }
      \edef\iabemaddresslist{\iabemaddresslist%
        \par%
        $^{\arabic{authorcounter}}$#3%
      }%
      \stepcounter{authorcounter}%
    }%
  \ignorespaces
}

\newenvironment{iabempaper}[3]{%
  \renewcommand{\iabemauthorlist}{}%
  \renewcommand{\iabememail}{}%
  \setcounter{authorcounter}{1}%
     #2
%  \twocolumn[
    \begin{center}
     \bfseries
     #1
     \bigskip

     \iabemauthorlist
     \mdseries
     \smallskip

     \iabemaddresslist
     \smallskip
 
     \iabememail
 
     \end{center}%
 % ]

}{%
}


% set up headers and footers
\pagestyle{fancy}
% \renewcommand{\chaptermark}[1]{\markboth{#1}{}}
% \renewcommand{\sectionmark}[1]{\markright{\thesection\ #1}}
\fancyhf{}
\fancyhead[LE,RO]{\bfseries}
\fancyhead[LO]{\bfseries IABEM 2018, Paris}
\fancyhead[RE]{\bfseries IABEM 2018, Paris}
\renewcommand{\headrulewidth}{0.5pt}
\renewcommand{\footrulewidth}{0.5pt}
\fancypagestyle{plain}{%
  \fancyhead{}
%  \renewcommand{\headrulewidth}{0pt}
}


% ------------------------------------------------------------------------
% --------------- DO NOT EDIT THE FILE ABOVE THIS LINE -------------------
% ------------------------------------------------------------------------

% Put additional \usepackage commands here. Use additional packages sparingly
% and not to change the overall layout of the paper.

% Add your own Macro definitions here.

\begin{document}

% Only the iabempaper environment should be contained between \begin{document}
% and \end{document}
 \begin{iabempaper}{%
% Put the paper tile here
Modeling Multiscale Interface Phenomena Using Nonlinear Transmission Conditions

}{%
% Put the list of authors here, using \iabemauthor or \iabemspeaker commands
  \iabemspeaker[jaydeep.p.bardhan@gsk.com]{Jaydeep P. Bardhan}{GlaxoSmithKline, Collegeville, PA, USA}
\iabemauthor{Thomas Klotz, Department of Computational and Applied Mathematics, Rice University, Houston, TX, USA}
  \iabemauthor{Ali Mehdizadeh Rahimi}{Department of Mechanical and Industrial Engineering, Northeastern University, Boston, MA, USA}
  \iabemauthor{Amirhossein Molavi Tabrizi}{Department of Physics, Northeastern University, Boston, MA, USA}
  \iabemauthor{Matthew G. Knepley}{Department of Computer Science, University of Buffalo, Buffalo, NY, USA}
} 

 

\bigskip


\noindent\textbf{Keywords:}
% describe the field with a few concise keywords
solvation, electrostatics, Poisson-Boltzmann, nonlinear, SLIC


\bigskip

% THE MAIN TEXT GOES HERE 

The atomic-scale structure of fluids at the solid-liquid interface
plays central roles in problems ranging from understanding proteins to
improving battery technology. This poses serious challenges for
quantitative modeling: on one hand, classical continuum models fail to
reproduce known facts even qualitatively correctly; on the other hand,
for many problems, atomistically detailed models are impractically
expensive. Most approaches for addressing this multiscale problem rely
either on complicated partial differential equation models, or on
coupling atomistic and continuum models. However, existing approaches
have failed to capture key nonlinear phenomena in the first layer of
fluid molecules. Using the electrostatic response of a liquid
surrounding a charged biomolecule as an example, we propose a new
approach, which capitalizes on the well-known fact that
boundary-integral equations focus attention on the interface itself,
and in particular on the transmission
conditions\cite{Bardhan14_asym,MolaviTabrizi16}. We have shown that
the transmission condition associated with the classical continuum
model, based on macroscopic dielectric theory, is easily corrected
with a simple nonlinear term that is a function of the local electric
field~\cite{Bardhan14_asym}.  In this talk, we will discuss solution
existence and uniqueness, as well as numerical methods. The corrected
model is easy to compute numerically on complicated geometries, as it
represents merely a diagonal perturbation of the usual BEM problem,
combined with a short nonlinear
iteration~\cite{Bardhan14_asym,MolaviTabrizi16}.  Results illustrate
that this remarkably simple correction to a familiar continuum model
increases accuracy to the level of fully atomistic calculations
thousands of times more expensive, and achieves this accuracy while
reducing the number of model parameters by an order of magnitude.  A
variety of related problems in interfacial response lead to modified
transmission conditions, and we suggest that the boundary-element
method community has a myriad of opportunities to advance multiscale
modeling.
 
\bigskip

 

\begin{thebibliography}{99}

% References go here, some examples are given  
%
\bibitem{Bardhan14_asym}
J.~P. Bardhan and M.~G. Knepley.
\newblock Modeling charge-sign asymmetric solvation free energies with
  nonlinear boundary conditions.
\newblock {\em J. Chem. Phys.}, 141:131103, 2014.

\bibitem{MolaviTabrizi16}
A.~Molavi Tabrizi, S.~Goossens, C.~D. Cooper, M.~G. Knepley, and
J.~P. Bardhan.
\newblock Extending the solvation-layer interface condition
  (SLIC) continuum electrostatic model to linearized
Poisson--Boltzmann solvent,
\newblock {\em Journal of Chemical Theory and Computation} 13:2897, 2017,.
\end{thebibliography}

% ------------------------------------------------------------------------
% --------------- DO NOT EDIT THE FILE BELOW THIS LINE -------------------
% ------------------------------------------------------------------------

\end{iabempaper}

\end{document}
